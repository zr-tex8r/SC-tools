%#!uplatex
\documentclass[uplatex,dvipdfmx,a4paper]{jsarticle}
\usepackage[hidelinks,pdfusetitle]{hyperref}
\usepackage{pxjahyper}
\usepackage[animate]{scnicefoot}
\usepackage{bxjalipsum}
\newcommand{\Pkg}[1]{\textsf{#1}}
%=================================================
\title{{\TeX}の脚注をナントカにする}
\author{某ZR(アレ)}
\date{アレコンフの日}
\begin{document}
\maketitle
%-------------------------------------------------
\section{\Pkg{scnicefoot}パッケージの使い方}

\begin{quote}\small\begin{verbatim}
\usepackage{scnicefoot}
\end{verbatim}\end{quote}

\verb|animate|オプションを付けるともっと素敵になる。
\begin{quote}\small\begin{verbatim}
\usepackage[animate]{scnicefoot}
\end{verbatim}\end{quote}

\section{使ってみた}

吾輩は猫である。\footnote{アヒルではない。}% 脚注してみる
名前はまだ無い。% 脚注の内容は非本質的なので出力されない.

どこで生れたかとんと見当がつかぬ。%
% 脚注で \verb を使っても大丈夫. (もちろん出力されない.)
\footnote{\verb|\expandafter|の使い方も見当がつかぬ。}%
何でも薄暗いじめじめした所でニャーニャー泣いていた事だけは
記憶している。

\section{minipage環境中の脚注}

\Pkg{scnicefoot}の既定では、{\LaTeX}標準の
「minipage環境中の脚注を別系統で扱いボックスの末尾に出力する」
という挙動が抑止され、
minipage中の脚注も外の脚注と全く同じように扱われる。
\footnote{もちろん、内容は非本質的なので出力されない。}

例えば→
\begin{minipage}{15zw}
智に働けば角が立つ。\\
情に棹させば流される。\footnote{フツーの脚注。}\\
意地を通せば窮屈だ。\footnote{スゴイ脚注。}\\
とかくに人の世は住みにくい。\footnote{トッテモスゴイ脚注!}
\end{minipage}

\section{ちなみに}
本文に脚注があるかどうかも本質的でないため、
\verb|\footnote|命令を使っていなくても大抵のページには
本質的な脚注が配置されます。
(白ページには出ないはず。)

\section{以下は「吾輩は猫である」のテキスト}

\jalipsum{wagahai}

\end{document}
